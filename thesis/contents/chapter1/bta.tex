\subsection{Анализ времени связывания}

Анализ времени связывания разделяет программные конструкции на домены согласно моментам, когда конкретная конструкция получила связывание.
В зависимости от цели применения могут выбираться разные домены времен связывания.

В данной работе цель --- указать порядок, в котором имена связываются со значениями.

\subsubsection{Обзор существующих решений}

Анализ времени связывания часто используется при offline-специализации программ~\cite{jones1993partial}. 
В этом случае он используется для определения того, какие данные известны статически и должны быть учтены при специализации, а какие неизвестны. 
Также часто определяется, какие функции вообще следует специализировать и каким образом. 

Анализ времени связывания существует для логического языка \prolog{}~\cite{leuschel2004prolog} и функционального-логического языка \mercury{}~\cite{vanhoof2004binding} --- представителей родственных реляционному программированию парадигм.
Однако, в языке \mercury{} анализ времени связывания~\cite{vanhoof2004binding} используется для эффективной компиляции. 
При этом используются только аннотации in и out --- статические и динамические переменные. 
Этого недостаточно, чтобы определить порядок вычислений при трансляции в функциональный язык.
Определение порядка вычислений в \mercury{} осуществляется во время более трудоемкого анализа модов (mode analysis), не существующего для \miniKanren{}. 
При этом непосредственное использование этого подхода для \miniKanren{} невозможно, так как не все языки семейства типизируемы, а анализ времени связывания \mercury{} осуществляется с учётом графа типов, построенного по программе. 

Система \logen{} реализует анализ времени связывания для чистого подмножества \prolog{}~\cite{leuschel2004prolog}.
Основное предназначение анализа в этой работе --- улучшение качества специализации, упорядочивания вызовов не производится. 

Работа~\cite{Thiemann1997AUF} описывает анализ времени связывания для лямбда-исчисления с функциями высшего порядка. 
Его цель также в том, чтобы определить порядок связывания переменных, поэтому авторы используют отрезок натурального ряда $\{ 0, 1, \dots, N\}$. 
Эта идея была использована в данной работе.

\section*{Заключение}

\paragraph{Целью} данной работы было создание транслятора реляционного языка в функциональный, способного в транслированной функции сохранить семантику исходного отношения в выбранном направлении.

Для достижения этой цели было поставлено несколько задач, каждая из которых была решена.

\begin{itemize}
    \item Разработка алгоритма транслирования абстрактного синтаксиса реляционного языка в абстрактный синтаксис функционального языка.

    Рассмотрены особенности \miniKanren{} и особенности трансляции. С их учётом создан абстрактный синтаксис функционального языка и разработан алгоритм трансляции. Реализация написана на \haskell{}. Полученный алгоритм способен транслировать программы на \miniKanren{} во всех направлениях за исключением, когда все переменные являются выходными. Доказано сохранение семантики при трансляции.

    \item Разработка алгоритма аннотирования, позволяющего транслятору определять направления и порядок вычислений конъюнктов.
    
    Для программ на \miniKanren{} введено понятие нормальной формы. На основе идеи анализа времени связывания разработан алгоритм аннотирования переменных для нормализованных программ. Рассмотрены способы приведения любых программ в нормальную форму как часть алгоритма аннотирования. Его реализация написана на \haskell{}. Доказана корректность.

    \item Тестирование и анализ результатов.
    
    Предложено несколько классификаций программ на \miniKanren{} в соответствии с проблемами, возникшими при создании алгоритмов аннотирования и трансляции. Создана база программ на \miniKanren{}, покрывающая предложенные классификации. Для тестирования результата трансляции в конкретном синтаксисе создан конкретный синтаксис \miniKanren{} (грамматика в приложении) и реализован его парсер, а так же транслятор абстрактного функционального синтаксиса в конкретный. Оба алгоритма реализованы на \haskell{}.
    
\end{itemize}

В рамках данной работы получены следующие результаты:
\begin{itemize}
    \item Разработан алгоритм трансляции реляционных программ в функциональные для конкретного направления с сохранением семантики транслируемых отношений;
    \item По результатам тестирования можно утверждать, что данный алгоритм работает для всех выделенных типов программ \miniKanren{}; ограничением трансляции является невозможность трансляции отношений на направлении, когда все аргументы являются выходными;
    \item Исходный код проекта можно найти на сайте~\ref{https://github.com/Pluralia/uKanren_translator/tree/master}, автор принимал участи под учётной записью \emph{Pluralia};
    \item Результаты работы опубликованы в сборнике конференции SEIM'20 и приняты на конференцию TEASE-LP'20.
\end{itemize}
\section{Разработка транслятора}
\label{translator}

Этот раздел посвящен разработке алгоритма трансляции \miniKanren{} в абстрактный функциональный язык программирования.
В первой части рассказывается об особенностях \miniKanren{} и способах транслировать программу с этими особенностями.
Вторая часть посвящена построению абстрактного синтаксическое дерева функционального языка с учётом особенностей трансляции \miniKanren{}.
Общее описание алгоритма трансляции находится в третьей части.
Четвёртая часть посвящена доказательству сохранения семантики программы на \miniKanren{} в наборе функций, полученных после работы алгоритма трансляции.

В нескольких частях примеры отсылаются к одному и тому же отношению $append^o$ (см. рисунок~\ref{lst:appendo}).
Оно связывает три списка, первые два из которых являются конкатенацией третьего.

\begin{figure}[h!]
  \begin{center}
  \begin{minipage}{0.4\textwidth}
  \begin{lstlisting}[language=Haskell, frame=single, numbers=left,numberstyle=\small, firstnumber=1, escapechar=|]
  $append^o$ $x$ $y$ $z$ =
    ($x$ $\equiv$ [] $\wedge$ $y$ $\equiv$ $z$) $\vee$ |\label{line:appendo2}|
    (fresh [h, t, r] (
        $x$ $\equiv$ $h$ : $t$ $\wedge$ |\label{line:appendo4}|
        $z$ $\equiv$ $h$ : $r$ $\wedge$ |\label{line:appendo5}|
        $append^o$ $t$ $y$ $r$ |\label{line:appendo6}|
    ))
    \end{lstlisting}
  \end{minipage}
  \end{center}
  \caption{Отнощение $append^o$}
  \label{lst:appendo}
\end{figure}

\subsection{Особенности \miniKanren{}}

\miniKanren{} является реляционным языком программирования, а то время как трансляция осуществляется в функциональный.
Данный раздел рассматривает особенности \miniKanren{} и способы их поддержания в функциональном языке.
Так как для тестирования абстрактный синтаксис функционального языка транслируется в конкретный синтаксис \haskell{}, описания решения проблем будут в терминах конструкций \haskell{}.
Однако, это не умаляет общности получаемого транслятора.
Используемые конструкции имеют аналоги в любом функциональном языке программирования, и можно написать транслятор абстрактного синтаксиса в любой конкретный.

Все рассматриваемые в данном разделе примеры являются результатом трансляции отношения $append^o$ в различных направлениях.
Само отношение $append^o$ на \miniKanren{} представлено на рисунке~\ref{lst:appendo}.

\begin{figure}[h!]
  \begin{center}
  \begin{minipage}{0.4\textwidth}
  \begin{lstlisting}[language=Haskell, frame=single, numbers=left,numberstyle=\small, firstnumber=8, escapechar=|]
  $append^o$ $x^1$ $y^1$ $z^0$ =
    ($x^1$ $\equiv$ [] $\wedge$               |\label{line:appendoOOIANN2}|
     $y^1$ $\equiv$ $z^0$) $\vee$             |\label{line:appendoOOIANN3}|
    (fresh [h, t, r] (
        $x^3$ $\equiv$ $h^1$ : $t^2$ $\wedge$ |\label{line:appendoOOIANN4}|
        $z^0$ $\equiv$ $h^1$ : $r^1$ $\wedge$ |\label{line:appendoOOIANN5}|
        $append^o$ $t^2$ $y^2$ $r^1$          |\label{line:appendoOOIANN6}|
    ))
    \end{lstlisting}
  \end{minipage}
  \end{center}
  \caption{Проаннотированное в обратном направлении отношение $append^o$}
  \label{lst:appendoOOIANN}
\end{figure}

%%%%%%%%%%%%%%%%%%%%%%%%%%%%%%%%%%%%%%%%%%%%%%%%%%%%%%%%%%%%%%%%%%%%%%%%%%%%%%%%

\subsubsection{Несколько выходных переменных}

Не всегда результатом выполнения отношения является единственный ответ.
Например, при выполнении отношения $append^o$ в обратном направлении, \miniKanren{} вычислит \emph{все} возможные \emph{пары} списков, дающие при конкатенации~$z$.

В~общем случае отношению $R \subseteq X_0 \times \dots \times X_n$, с известными аргументами $X_{i_0}, \dots X_{i_k}$, и неизвестными $X_{j_0}, \dots X_{j_l}$, соответствует функция, возвращающая список результатов $F : X_{i_0} \to \dots \to X_{i_k} \to [X_{j_0} \times \dots \times X_{j_l}]$. 

В качестве решения проблемы предлагается использовать кортежи.
Пример трансляции $append^o$ в обратном направлении приведён на рисунке~\ref{lst:appendoOOITR}.
$OOI$ рядом с названием функции обозначает направление трансляции отношения.
Так, $O$ --- output и $I$ --- input.
Пример получения кортежа в качестве результата находится в строке~\ref{line:appendoOOITR7}.

\begin{figure}[h!]
  \begin{center}
  \begin{minipage}{0.7\textwidth}
  \begin{lstlisting}[language=Haskell, frame=single, numbers=left,numberstyle=\small, firstnumber=1, escapechar=|]
    appendoOOI x0 = appendoOOI0 x0 ++ appendoOOI1 x0 |\label{line:appendoOOITR1}|
    appendoOOI0 s4$@$s0 = do                         |\label{line:appendoOOITR2}|
      let s3 = []                                    |\label{line:appendoOOITR3}|
      return $\$$ (s3, s0)                           |\label{line:appendoOOITR4}|
    appendoOOI0 _ = []                               |\label{line:appendoOOITR5}|
    appendoOOI1 s4$@$(s5 : s7) = do                  |\label{line:appendoOOITR6}|
      (s6, s0) <- appendoOOI s7                      |\label{line:appendoOOITR7}|
      let s3 = (s5 : s6)                             |\label{line:appendoOOITR8}|
      return $\$$ (s3, s0)                           |\label{line:appendoOOITR9}|
    appendoOOI1 _ = []                               |\label{line:appendoOOITR10}|
    \end{lstlisting}
  \end{minipage}
  \end{center}
  \caption{Результат трансляции отношения $append^o$ в обратном направлении}
  \label{lst:appendoOOITR}
\end{figure}

%%%%%%%%%%%%%%%%%%%%%%%%%%%%%%%%%%%%%%%%%%%%%%%%%%%%%%%%%%%%%%%%%%%%%%%%%%%%%%%%

\subsubsection{Пересечение результатов дизъюнктов}

Дизъюнкты в программе на \miniKanren{} независимы, то есть все ответы из каждого дизъюнкта объединяются для получения результата выполнения отношения.
Чтобы поддержать данную особенность, будем транслировать каждый дизъюнкт во вспомогательную функцию.
Самому отношению будет соответствовать функция, конкатенирующая результаты этих вспомогательных функций.
Стоит обратить внимание, что рекурсивно вызывается функция $appendoOOI$, построенная по всему отношению, а не какие-либо вспомогательные функции.

Примером такого поведения и соответствующей трансляции является $append^o$ в обратном направлении (см. рисунок~\ref{lst:appendoOOITR}.
Здесь $appendoOOI$ является основной функцией, объединяющей результаты вспомогательных функций $appendoOOI0$ и $appendoOOI1$.
Если запустить $appendoOOI$ на списке из трёх элементов $[1,~2,~3]$, можно получить список всех пар списков, конкатенация которых даёт входящий список: \\ $[([],~[1,~2,~3]),~([1],~[2,~3]),~([1,~2],~[3]),~([1,~2,~3],~[])]$.
Анализируя этот ответ легко понять, что первый кортеж получен из первого дизъюнкта (функции $appendoOOI0$), а все последующие --- из второго (функции $appendoOOI1$).

%%%%%%%%%%%%%%%%%%%%%%%%%%%%%%%%%%%%%%%%%%%%%%%%%%%%%%%%%%%%%%%%%%%%%%%%%%%%%%%%

\subsubsection{Недетерминированность результатов}

\miniKanren{} способен выдавать несколько вариантов одной переменной в качестве результата, создавая недетерминированность.
Монада списка --- способ выразить недетерминированность в функциональном языке.
Используем её в трансляции.
Кроме того, в \haskell{} для неё есть удобная нотация, называемая \emph{do-нотацией}\footnote{Описание do-нотации языка \haskell{}: \url{https://en.wikibooks.org/wiki/Haskell/do\_notation}, дата последнего посещения: 14.05.2020}.

Её применение так же можно видеть на примере трансляции $append^o$ в обратном направлении, представленном на рисунке~\ref{lst:appendoOOITR}.
Связывание в строке~\ref{line:appendoOOITR7} означает, что результат будет вычислен для каждого элемента списка, полученного из рекурсивного вызова функции.
Унификации неизвестных переменных (например $x~\equiv~[]$ и $x~\equiv~h~\%~t$) при трансляции преобразуются в $let$-связывания (строки~\ref{line:appendoOOITR3} и~\ref{line:appendoOOITR8}). 

%%%%%%%%%%%%%%%%%%%%%%%%%%%%%%%%%%%%%%%%%%%%%%%%%%%%%%%%%%%%%%%%%%%%%%%%%%%%%%%%

\subsubsection{Переменные, принимающие все возможные значения}

При вычислении отношений в различных направлениях нередко встречается ситуация, когда $fresh$-переменные, унифицирующиеся только друг с другом.
В этом случае они остаются свободными.
В \miniKanren{} такие переменные могут принимать все допустимые значения.

Данная проблема является проблемой и для анализа времени связывания, обсуждаемого в следующем разделе.
Результатом её разрешения является появление в дизъюнкте специальных унификаций вида $x~\equiv~<gen:>$, где $x$ --- переменная, оставшаяся свободной, а $<gen:>$ --- нотация, позволяющая транслятору понять необходимость генерации.
Таким образом, всё, что остаётся сделать на этапе трансляции --- правильно раскрыть нотацию и произвести генерацию.

Рассмотрим пример трансляции $append^o \ x \ ? \ ?$ (см. рисунок~\ref{lst:appendoIOOTR}).
Генерация здесь происходит в строке~\ref{line:appendoIOOTR3} и выглядит как вызов функции, возвращающий список.
Функция $gen$ являющейся функцией класса типов $Generator$.
Её реализация лежит на плечах пользователя: в зависимости от типа переменной она может генерировать списки различных сущностей в различном порядке.
Реализация $Generator$ для списков и цифр, а так же сам класс типов $Generator$ представлены на рисунке~\ref{lst:generator}.

\begin{figure}[h!]
  \begin{center}
  \begin{minipage}{0.8\textwidth}
  \begin{lstlisting}[language=Haskell, frame=single, numbers=left,numberstyle=\small, firstnumber=22, escapechar=|]
        appendoIOO x0 = appendoIOO0 x0 ++ appendoIOO1 x0
        appendoIOO0 s0$@$[] = do
          s1 <- (gen)                                    |\label{line:appendoIOOTR3}|
          let s2 = s1                                    
          return $\$$ (s1, s2)
        appendoIOO0 _ = []
        appendoIOO1 s0$@$(s3 : s4) = do
          (s1, s5) <- appendoIOO s4
          let s2 = (s3 : s5)
          return $\$$ (s1, s2)
        appendoIOO1 _ = []
    \end{lstlisting}
  \end{minipage}
  \end{center}
  \caption{Результат трансляции отношения $append^o \ x \ ? \ ?$}
  \label{lst:appendoIOOTR}
\end{figure}

\begin{figure}[h!]
  \begin{center}
  \begin{minipage}{0.7\textwidth}
  \begin{lstlisting}[language=Haskell, frame=single, numbers=left,numberstyle=\small, firstnumber=11, escapechar=|]
        class Generator a where
          gen :: [a]
        
        instance (Generator a) => Generator [a] where
          gen = [] : do
            xs <- gen
            x <- gen
            return (x : xs)
        
        instance Generator Int where
          gen = [0..9]
    \end{lstlisting}
  \end{minipage}
  \end{center}
  \caption{Класс типов $Generator$ и его реализации для списков и цифр}
  \label{lst:generator}
\end{figure}

%%%%%%%%%%%%%%%%%%%%%%%%%%%%%%%%%%%%%%%%%%%%%%%%%%%%%%%%%%%%%%%%%%%%%%%%%%%%%%%%

\subsubsection{Порядок и направление при исполнении отношений}

Самая главная особенность --- порядок и направление исполнения отношений внутри целевого может отличается для разных направлений.
Так, отношение, выполненное в заданном направлении, можно рассматривать как функцию из известных аргументов в неизвестные. 
Например, отношение $append^o$, выполненное в прямом направлении, соответствует функции конкатенации списков $x$ и $y$.

Отношение $append^o$ состоит из двух дизъюнктов. 
Первый дизъюнкт означает, что если $x$ является пустым списком, то $y$ совпадает с $z$. 
Второй дизъюнкт означает, что $x$ и $z$ являются списками, начинающимися с одного и того же элемента, при этом хвостом $z$ является результат конкатенации хвоста списка $x$ со списком $y$. 
Унификация с участием неизвестной переменной $z$ указывает на то, \emph{как} вычислить её значение, в то время как унификация известной переменной $x$ --- \emph{при каком условии}.

Автоматическая трансляция $append^o$ в прямом направлении создаст функцию, приведенную на рисунке~\ref{lst:appendoIIOTR}.
В двух уравнениях первая переменная сопоставляется с образцом. 
В первом случае мы сразу возвращаем второй список как результат, в то время как во втором необходимо осуществить рекурсивный вызов построенной функции. 

\begin{figure}[h!]
  \begin{center}
  \begin{minipage}{0.7\textwidth}
  \begin{lstlisting}[language=Haskell, frame=single, numbers=left,numberstyle=\small, firstnumber=8, escapechar=|]
        appendoIIO x0 x1 = appendoIIO0 x0 x1 ++ appendoIIO1 x0 x1
        appendoIIO0 s3$@$[] s0 = do                                  |\label{line:appendoIIOTR2}|
          let s4 = s0
          return $\$$ (s4)
        appendoIIO0 _ _ = []                                         |\label{line:appendoIIOTR5}|
        appendoIIO1 s3$@$(s5 : s6) s0 = do                           |\label{line:appendoIIOTR6}|
          (s7) <- appendoIIO s6 s0
          let s4 = (s5 : s7)
          return $\$$ (s4)
        appendoIIO1 _ _ = []                                         |\label{line:appendoIIOTR10}|
  \end{lstlisting}
  \end{minipage}
  \end{center}
  \caption{Результат трансляции отношения $append^o$ в прямом направлении}
  \label{lst:appendoIIOTR}
\end{figure}

Пример трансляции $append^o$ в обратном направлении приведен на рисунке~\ref{lst:appendoOOITR}.

Нетрудно заметить, что порядок вычислений в функциях не совпадает с порядком конъюнктов в исходном отношении. 
Например, рекурсивный вызов отношения $append^o$ производится в последнем конъюнкте (см. рис.~\ref{lst:appendo}, строка~\ref{line:appendo6}), в то время как в функциях выполняется в первую очередь.

Данная проблема была указана во введении как мотивация к разработке алгоритма аннотирования переменных.
Адаптация анализа времени связывания к \miniKanren{} и есть решение данной проблемы (рассматривается в следующем разделе).
Рассмотрим конкретный пример, как аннотации переменных помогают выбрать порядок и направления вычислений отношений.

Пусть есть проаннотированное в обратном направлении отношение $append^o$ (см. рисунок~\ref{lst:appendoOOIANN}), которое необходимо транслировать.
Результат трансляции есть на рисунке~\ref{lst:appendoOOITR}.

Рассмотрим первый дизъюнкт.
В начале определим направления вычислений каждой из них.
Направление первой из них (в строка~\ref{line:appendoOOIANN2}): $let~x~=~[]$, так как аннотация константы всегда меньше аннотации переменной.
Направление второй --- $let~y~=~z$, потому что аннотация $z$ является $0$ (входная переменная), в то время как $y$ --- $1$.
Таким образом, направление выбирается в соответствии с тем, какой части унификации принадлежит большая аннотация: кто содержит наибольшую, тому будет происходить присваивание.
Теперь определим порядок.
Для этого достаточно отсортировать получившиеся на прошлом шаге определения от меньшего к большему по аннотации определяемой переменной.
В примере аннотации $x$ и $y$ совпадают и равны $1$, поэтому в данном случае нам не важен их порядок.
При трансляции (см. рисунок~\ref{lst:appendoOOITR}) определение $y$ станет частью сопоставления с образцом (о том, как это работает, будет рассказано в следующем разделе) в строке~\ref{line:appendoOOITR2}, а определение $x$ --- строкой~\ref{line:appendoOOITR3}.

Перейдём ко второму дизъюнкту.
Определим направления.
Унификация в строке~\ref{line:appendoOOIANN4} обратится в определение $let~x~=~(h~:~t)$, потому что аннотация $x$ больше обоих аннотаций переменных $h$ и $t$.
Строка~\ref{line:appendoOOIANN5} даст определение $let~(h~:~r)~=~z$.
Направление вызова функции в строке~\ref{line:appendoOOIANN6} также определим по максимальной аннотации аргументов вызова.
В данном случае она равна $2$ и встречается у двух переменных --- $t$ и $y$.
Это означает, что вызов происходит в том же направлении, что и исходное отношение, и направление будет выглядеть так: $(t,~y) \leftarrow append^oOOI~r$.
Сортировка определений позволит получить их следующий порядок: $(h~:~r)$, $(t,~y)$, $x$.
При трансляции они окажется, соответственно, в~\ref{line:appendoOOITR6},~\ref{line:appendoOOITR7}~и~\ref{line:appendoOOITR8} строках.

\subsection{Построение абстрактного синтаксического дерева функционального языка}
\label{lab:ast}

В предыдущем разделе были разобраны особенности трансляции, связанные с \miniKanren{}.
В этом разделе рассмотрены особенности, проявившиеся в процессе разработки алгоритма трансляции и повлиявшие на структуру абстрактного синтаксическое дерева функционального языка.

%%%%%%%%%%%%%%%%%%%%%%%%%%%%%%%%%%%%%%%%%%%%%%%%%%%%%%%%%%%%%%%%%%%%%%%%%%%%%%%%

\subsubsection{Сопоставление с образцом для входных переменных}

Сопоставление с образцом --- хороший способ отфильтровать заведомо ложные вычисления, используя информацию о типе конструктора аргумента.
В качестве примера рассмотрим трансляцию $append^o$ в прямом направлении (см. рисунок~\ref{lst:appendoIIOTR}).
Дизъюнкты исходного отношения $append^o$ (см. рисунок~\ref{lst:appendo}) содержат унификации первого аргумента, являющегося входным: первый дизъюнкт --- унификацию с пустым списком (строка~\ref{line:appendoOOIANN2}), второй --- с непустым (строка~\ref{line:appendoOOIANN4}).
При трансляции такие унификации превращаются в сопоставление с образцом (см. строки~\ref{line:appendoIIOTR2} и~\ref{line:appendoIIOTR6}).
Как результат --- если первый аргумент функции является пустым списком, то успешно вычислится только $appendoIIO0$.

Стоит отметить, зачем нужны строки~\ref{line:appendoIIOTR5} и~\ref{line:appendoIIOTR10}.
Они представляют собой сопоставление с образцом, которое всегда завершится успехом, однако, они не влияют на результат вычисления, так как возвращают пустой список.
$appendoIIO1$, в случае не успешного сопоставления с образцом в строке~\ref{line:appendoIIOTR6}, попытается найти уравнение, в котором сопоставление с образцом пройдёт успешно.
Если строки~\ref{line:appendoIIOTR10} не будет, то вычисление функции $appendoIIO1$ завершится ошибкой за отсутствием возможности обработать соответствующий вход.

В случае, если в отношении на \miniKanren{} одной входной переменной соответствовало несколько унификаций, для неё появляется возможность выбрать, какая из них станет сопоставлением с образцом.
Алгоритм выбирает ту из унификаций, которая обеспечит наибольшую вложенность конструкторов. 

%%%%%%%%%%%%%%%%%%%%%%%%%%%%%%%%%%%%%%%%%%%%%%%%%%%%%%%%%%%%%%%%%%%%%%%%%%%%%%%%

\subsubsection{Совпадение имён в сопоставлениях с образцом}

Имена переменных, используемых в сопоставлении с образцом, могут совпадать для разных аргументов.
Например, во втором дизъюнкте $append^o$ на рисунке~\ref{lst:appendo} есть унификации переменных $x$ и $z$.
Если мы будем транслировать $append^o \ x \ ? \ z$, то получим перекрытие имён переменных.
Переменная $h$, участвующая в обеих унификациях, окажется и в обоих сопоставлениях с образцом: для первого и третьего аргументов.

На рисунке~\ref{lst:appendoIOITR} представлен результат трансляции $append^o$ в обсуждаемом направлении.
Второму дизъюнкту в нем соответствует функция $appendoIOI1$, а переменной $h$ --- переменная $s3$.
Чтобы избежать перекрытия имён, $s3$ была переименована в $p2$ в сопоставлении с образцом для второго аргумента.
Переименовывание нарушило условие, созданное при трансляции в данном направлении: оба аргумента-списка должны иметь одинаковый первый элемент списка.
Восстановление этого условия происходит за счёт применения охранного выражения, проверяющего списки на равенство $|~s3~==~s2$.

\begin{figure}[h!]
  \begin{center}
  \begin{minipage}{0.8\textwidth}
  \begin{lstlisting}[language=Haskell, frame=single, numbers=left,numberstyle=\small, firstnumber=139, escapechar=|]
    appendoIOI x0 x1 = appendoIOI0 x0 x1 ++ appendoIOI1 x0 x1
    appendoIOI0 s0$@$[] s2@s1 = return $\$$ (s1)
    appendoIOI0 _ _ = []
    appendoIOI1 s0$@$(s3 : s4) s2$@$(p2 : s5) $|$ s3 == p2 = do
      (s1) <- appendoIOI s4 s5
      return $\$$ (s1)
    appendoIOI1 _ _ = []
    \end{lstlisting}
  \end{minipage}
  \end{center}
  \caption{Результат трансляции отношения $append^o \ x \ ? \ z$}
  \label{lst:appendoIOITR}
\end{figure}

Здесь же стоит заметить ещё одну особенность: рядом с каждым сопоставлением с образцом находится его ``псевдоним''.
Этот псевдоним --- исходное имя входной переменной до замены на сопоставление с образцом.
Его необходимо сохранить для случая, если внутри тела функции потребуется именно эта переменная, а не переменные из сопоставления с образцом.

%%%%%%%%%%%%%%%%%%%%%%%%%%%%%%%%%%%%%%%%%%%%%%%%%%%%%%%%%%%%%%%%%%%%%%%%%%%%%%%%

\subsubsection{Совпадение имен в определениях}

Переменные в определениях так же могут совпадать, однако, для них нельзя использовать охранные выражения.
Для решения проблемы совпадения имен в определениях будем использовать ветвление.

На рисунке~\ref{lst:appendoAssign} представлено модифицированное отношение $apppend^o$ (см. рисунок~\ref{lst:appendo}).
Оно связывает три списка таких, что первый является повтором первого элемента второго списка, а третий --- конкатенацией первого и второго списков.

\begin{figure}[h!]
  \begin{center}
  \begin{minipage}{0.45\textwidth}
  \begin{lstlisting}[language=Haskell, frame=single, numbers=left,numberstyle=\small, firstnumber=146, escapechar=|]
  $append^oAssign$ $x$ $y$ $z$ =
    ($x$ $\equiv$ [] $\wedge$                     |\label{line:appendoAssign2}|
     $y$ $\equiv$ $z$) $\vee$                     |\label{line:appendoAssign3}|
    ($fresh$ [$h$, $t$, $r$, $p$, $ps$, $c$, $cs$] (
        $x$ $\equiv$ $h$ : $t$ $\wedge$           |\label{line:appendoAssign5}|
        $z$ $\equiv$ $h$ : $r$ $\wedge$           |\label{line:appendoAssign6}|
        $z$ $\equiv$ $p$ : ($p$ : $ps$) $\wedge$ |\label{line:appendoAssign7}|
        $z$ $\equiv$ $c$ : ($c$ : $cs$) $\wedge$ |\label{line:appendoAssign8}|
        $append^oAssign$ $t$ $y$ $r$              |\label{line:appendoAssign9}|
    ))
    \end{lstlisting}
  \end{minipage}
  \end{center}
  \caption{Отношение $append^oAssign$}
  \label{lst:appendoAssign}
\end{figure}

Рассмотрим результат трансляции отношения $append^oAssign \ ? \ ? \ z$, представленный на рисунке~\ref{lst:appendoAssignOOITR}.

\begin{figure}[h!]
  \begin{center}
  \begin{minipage}{0.7\textwidth}
  \begin{lstlisting}[language=Haskell, frame=single, numbers=left,numberstyle=\small, firstnumber=156, escapechar=|]
 appendoAssignOOI x0 = appendoAssignOOI0 x0 ++ appendoAssignOOI1 x0
 appendoAssignOOI0 s2$@$s1 = do
   let s0 = []
   return $\$$ (s0, s1)
 appendoAssignOOI0 _ = []
 appendoAssignOOI1 s2$@$(s8 : (p2 : s9)) $|$ s8 == p2 = do
   let (s3 : s5) = s2
   let (s6 : (c4 : s7)) = s2                 |\label{line:appendoAssignOOITR8}|
   (s4, s1) <- appendoAssignOOI s5
   let s0 = (s3 : s4)
   if (s6 == c4) then return $\$$ (s0, s1) else [] |\label{line:appendoAssignOOITR11}|
 appendoAssignOOI1 _ = []
    \end{lstlisting}
  \end{minipage}
  \end{center}
  \caption{Результат трансляции отношения $append^oAssign \ ? \ ? \ z$}
  \label{lst:appendoAssignOOITR}
\end{figure}

Строка~\ref{line:appendoAssignOOITR8} содержит определение, полученное при трансляции унификации $z~\equiv~c~:~(c~:~cs)$.
Переменная $c$ здесь стала переменной $s6$ и получили определение $let~(s6~:~(s6~:~s7))~=~s2$.
Как и в случае аналогичной проблемы с сопоставлением с образцом, переименуем повторившуюая переменную.
После чего необходимо добавить проверку на равенство исходной и переменной-замены.
Все такие проверки накапливаются и происходят в конце --- перед возвратом значения.
Так, в~\ref{line:appendoAssignOOITR11} строке показано, что в случае невыполнения условия необходимо вернуть пустой список.
Если условие выполняется, то возвращается результат.

%%%%%%%%%%%%%%%%%%%%%%%%%%%%%%%%%%%%%%%%%%%%%%%%%%%%%%%%%%%%%%%%%%%%%%%%%%%%%%%%

\subsubsection{Трансляция конструкторов}

Терм \miniKanren{} может быть произвольным конструктором.
В этом случае, чтобы успешно выполнить полученную после трансляции функцию, необходимо знать, как вычислять данный конструктор.
При трансляции будем считать, что пользователь сам позаботится о способе вычисления конструктора, определив соответствующий тип.
На рисунке~\ref{lst:peano} приведен один из таких пользовательских конструкторов, реализующих натуральные числа.

\begin{figure}[h!]
  \begin{center}
  \begin{minipage}{0.4\textwidth}
  \begin{lstlisting}[language=Haskell, frame=single, numbers=left,numberstyle=\small, firstnumber=168, escapechar=|]
    data Peano = O $|$ S Peano
    \end{lstlisting}
  \end{minipage}
  \end{center}
  \caption{Тип данных $Peano$ --- определение конструкторов $O$ и $S$}
  \label{lst:peano}
\end{figure}

Тем не менее, в трансляторе поддержаны некоторые конструкторы, такие как конструкторы списков и булевы константы.

%%%%%%%%%%%%%%%%%%%%%%%%%%%%%%%%%%%%%%%%%%%%%%%%%%%%%%%%%%%%%%%%%%%%%%%%%%%%%%%%

\subsubsection{Абстрактный синтаксис функционального языка}

Абстрактный синтаксис используемого подмножества \haskell{} приведён на рисунке~\ref{lst:funcast}

\begin{figure}[h!]
  \begin{center}
  \begin{minipage}{0.75\textwidth}
  \begin{lstlisting}[language=Haskell, frame=single, numbers=left,numberstyle=\small, firstnumber=169, escapechar=|]
    data Atom = Var String
              $|$ Ctor String [Atom]
              $|$ Tuple [String]
    
    data Expr = Term Atom
              $|$ Call String [Atom]
    
    data Assign = Assign Atom Expr
    
    newtype Guard = Guard [Atom]
    
    data Pat = Pat (Maybe String) Atom
    
    data Line = Line [Pat] [Guard] [Assign] [Guard] Expr
    
    data F = F String [Line]
    
    newtype FuncProgram = FuncProgram [F]
    \end{lstlisting}
  \end{minipage}
  \end{center}
  \caption{Абстрактный синтаксис функционального языка}
  \label{lst:funcast}
\end{figure}

Транслированная программа $FuncProgram$ представляет собой множество функций $F$.

Каждая функция $F$ определяется именем и списком вспомогательных функций $Line$.

Каждая вспомогательная функция $Line$ состоит из списка сопоставлений с образцом $Pat$ (представляющего собой список аргументов), списка охранных выражений $Guard$ для сопоставления с образцом, списка определений $Assign$, списка охранных выражений $Guard$ для определений и значения выражения.

Сопоставление с образцом $Pat$ состоит из опционального псевдонима и тела сопоставления с образцом, называемого в данном случае $Atom$.

$Atom$ является аналогом $Term$ из \miniKanren{} с небольшим расширением.
$Atom$ может быть переменной или конструктором, однако, ещё он может быть кортежем (конструктор $Tuple$).
Кореж --- список переменных без конструктора.
Используется, когда необходимо вернуть несколько переменных после вызова функции.

Охранное выражение $Guard$ --- список $Atom$, которые необходимо проверить друг с другом на равенство.

Определение $Assign$ представляет собой $Atom$ и $Expr$ --- значение выражения $Expr$ будет сопоставлено $Atom$.

Выражение $Expr$ может быть или тоже $Atom$, или вызовом функции на списке $Atom$.

\subsection{Алгоритм трансляции}

Данный раздел посвящен самому алгоритму трансляции.
Первая часть вводит абстрактный синтаксис в соответствии со всеми особенностями, затронутыми в двух предыдущих разделах.
Вторая часть приводит общий алгоритм трансляции программы в абстрактном синтаксисе \miniKanren{} в программу в абстрактном синтаксисе функционального языка.

%%%%%%%%%%%%%%%%%%%%%%%%%%%%%%%%%%%%%%%%%%%%%%%%%%%%%%%%%%%%%%%%%%%%%%%%%%%%%%%%

\subsubsection{Абстрактный синтаксис функционального языка}

\begin{figure}[h!]
  \begin{center}
  \begin{minipage}{0.7\textwidth}
  \begin{lstlisting}[language=Haskell, frame=single, numbers=left,numberstyle=\small, firstnumber=30, escapechar=|]
        data Atom = Var String
                  $|$ Ctor String [Atom]
                  $|$ Tuple [String]
        
        data Expr = Term Atom
                  $|$ Call String [Atom]
        
        data Assign = Assign Atom Expr
        
        newtype Guard = Guard [Atom]
        
        data Pat = Pat (Maybe String) Atom
        
        data Line = Line [Pat] [Guard] [Assign] [Guard] Expr
        
        data F = F String [Line]
        
        newtype FuncProgram = FuncProgram [F]
    \end{lstlisting}
  \end{minipage}
  \end{center}
  \caption{Абстрактный синтаксис функционального языка}
  \label{lst:funcast}
\end{figure}

Транслированная программа $FuncProgram$ представляет собой множество функций $F$.

Каждая функция $F$ определяется именем и списком вспомогательных функций $Line$.

Каждая вспомогательная функция $Line$ состоит из списка сопоставлений с образцом $Pat$ (представляющего собой список аргументов), списка охранных выражений $Guard$ для сопоставления с образцом, списка определений $Assign$, списка охранных выражений $Guard$ для определений и значения выражения.

Сопоставление с образцом $Pat$ состоит из опционального псевдонима и тела сопоставления с образцом, называемого в данном случае $Atom$.

$Atom$ является аналогом $Term$ из \miniKanren{} с небольшим расширением.
$Atom$ может быть переменной или конструктором, однако, ещё он может быть кортежем (конструктор $Tuple$).
Кореж --- список переменных без конструктора.
Используется, когда необходимо вернуть несколько переменных после вызова функции.

Охранное выражение $Guard$ --- список $Atom$, которые необходимо проверить друг с другом на равенство.

Определение $Assign$ представляет собой $Atom$ и $Expr$ --- значение выражения $Expr$ будет сопоставлено $Atom$.

Выражение $Expr$ может быть или тоже $Atom$, или вызовом функции на списке $Atom$.

%%%%%%%%%%%%%%%%%%%%%%%%%%%%%%%%%%%%%%%%%%%%%%%%%%%%%%%%%%%%%%%%%%%%%%%%%%%%%%%%

\subsubsection{Алгоритм трансляции}

Первым шагом алгоритма трансляции, безусловно, будет запуск алгоритма аннотирования со всеми нормализациями.
Таким образом, из произвольной программы на \miniKanren{} будет получен стек всех вызовов этой программы, каждый вызов в котором --- нормализованное проаннотированное в определённом направлении определение на \miniKanren{}.

Далее для каждого дизъюнкта каждого определения со стека вызовов запускается сам алгоритм трансляции.
В первом приближении он состоит из следующих шагов:
\begin{itemize}
    \item Формирование списков входных и выходных переменных $in$ и $out$;
    \item Разбиение конъюнктов на те, которые могут стать сопоставлениями с образцом и все остальные;
    \item Удаление перекрытий имён в сопоставлении с образцом путём формирования охранных выражений;
    \item Получение направлений конъюнктов;
    \item Получение порядка определений;
    \item Удаление перекрытий имён в определениях путём формирования условий для ветвления;
\end{itemize}

Множество из этих шагов должны быть очевидны из рассмотренных в предыдущих разделах особоенностей \miniKanren{} и трансляции.
Подробнее рассмотрим только два шага --- получение направлений конъюнктов и получение порядка определений.

Получение направлений конъюнктов происходит по-разному для унификаций и вызовов.
Если конъюнкт --- унификация, определим значение максимальной аннотации в каждой из её частей.
Теперь возможны три случая:
\begin{itemize}
    \item Если эти значения равны, то унификация становится условием ветвления;
    \item Если значение левой части больше, то унификация превращается в определение, где левая часть зависит от правой; в ответ сохраняем не только получившееся определение, но и максимальную аннотацию левой части --- это необходимо для получения порядка вычисления определений;
    \item Оставшийся случай симметричен;
\end{itemize}

Если конъюнкт --- вызов отношения, определим его выходные переменные.
Для этого достаточно узнать максимальную аннотацию его аргументов --- переменные обладающие таковой, стали известны после выполнения вызова и являются зависимыми.
Так же, как делали для унификаций, сохраним значение максимальной аннотации.

Получим порядок определений.
Для этого отсортируем их по максимальной аннотации их зависимой части.

\subsection{Корректность алгоритма}

Трансляция считается корректной, если семантика полученной после трансляции функции совпадает с семантикой исходного отношения в заданном направлении.
В целом, для этого необходимо перебрать все программы на \miniKanren{}, применить к ним алгоритм трансляции и проверить, изменилось ли семантика.
Однако, перебор все программ --- алгоритмически неразрешимая задача.

Для доказательства корректности предлагается сделать следующее:
\begin{itemize}
    \item Рассмотреть корректность перенесения особенностей \miniKanren{} в функциональную парадигму;
    \item Проанализировать семантику конструктов \miniKanren{} в терминах функциональной парадигмы;
    \item Проверить, что конструкты функционального языка, полученные из конструктов \miniKanren{}, обладают той же семантикой;
\end{itemize}

Такой подход не гарантирует полностью корректный транслятор (что сделать невозможно), но позволит говорить о его корректности в первом приближении.

В разделе "Особенности \miniKanren{} и способы их трансляции" рассмотрены способы транслировать особенности \miniKanren{} в функциональную парадигму.
Корректность этих способов вытекает из их определения.

Проанализируем, чем являются конструкты нормализованной программы на \miniKanren{} для функциональной парадигмы:
\begin{itemize}
    \item Fresh --- объявление и определение переменной;
    \item Унификация или вызов отношения на полностью известных переменных --- предикат, проверка соответствия переменных;
    \item Унификация --- переопределение переменной: считаем, что при fresh переменная объявляется и получает своё значение --- все возможные значения её типа;
    \item Вызов отношения --- вызов функции, возвращает результат;
    \item Конъюнкт --- переопределение или предикат;
    \item Дизъюнкт --- функция, скоуп: внутри конъюнкции все конъюнкты, представляющие собой переопределения, могут использовать определённые ими переменные;
    \item Тело отношения --- запуск нескольких функций и объединение их результатов;
\end{itemize}

Проанализируем результат трансляции для каждого их этих конструктов и проверим сохранение семантики:
\begin{itemize}
    \item Унификация или вызов отношения на всех известных переменных --- $if$ или $guard$;
    \item Fresh + Унификация --- сопоставление с образом или определение;
    \item Fresh + Вызов отношения --- определение;
    \item Конъюнкт --- определение, $if$ или $guard$;
    \item Дизъюнкт --- вспомогательная функция;
    \item Тело отношения --- конкатенация результатов вызовов вспомогательных функций;
\end{itemize}

Таким образом мы показали, что для описанных конструктов семантика сохраняется.
Однако, мы не рассмотрели случай вызовов отношений, где все аргументы являются выходными.
Такие программы запрещены для трансляции и на текущий момент являются ограничением подхода.

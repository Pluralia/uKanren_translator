\subsection{Понятие нормальной формы}
\label{lab:normform}

\emph{Нормальная форма отношения на \miniKanren{}} --- представление отношения на \miniKanren{}, в котором все свободные переменные введены в область видимости на самом верхнем уровне; цель --- дизъюнкцию конъюнкций вызовов отношений или унификаций термов; отсутствуют унификации двух конструкторов.
Соответствующий абстрактный синтаксис приведен на рисунке~\ref{fig:normMiniKanren}.

\begin{figure}[h!]
    \begin{center}
    \begin{minipage}{0.5\textwidth}
    \begin{align*}
      Goal  &: \underline{fresh} \ [Name] \ (\bigvee \bigwedge Goal') \\
      Goal' &: \underline{call} \ Name \ [Var] \\
            &\mid Var \equiv Term \\
      Term  &: Var \\ 
            &\mid \underline{cons} \ Name \ [Term]
    \end{align*}
    \end{minipage}
    \end{center}
  \caption{Абстрактный синтаксис нормализованной программы на \miniKanren{}}
  \label{fig:normMiniKanren}
\end{figure}

\emph{Нормальная форма программы на \miniKanren{}} --- представление программы на \miniKanren{}, при котором цель программы является нормализованным отношением; цель каждого определения является нормализованным отношением. 

Важно заметить, что любое отношение \miniKanren{} можно преобразовать в нормальную форму.
Доказательство данного утверждения приведено в разделе~\ref{lab:normProof}.

Рассмотрим отличия нормализованной программы от ненормализованной.
\begin{itemize}
    \item Тело определения находится в дизъюнктивной нормальной форме.
    \item Все свободные переменные введены при помощи $fresh$ на самом верхнем уровне.
    \item Не существует унификаций термов-конструкторов.
    \item Не существует вызовов с аргументами-конструкторами.
\end{itemize}

Такие ограничения вводятся с целью упрощения процесса аннотирования и трансляции в целом.
\begin{itemize}
    \item ДНФ тела позволяет уменьшить глубину вложенности программы.
    \item $fresh$-цель задаёт область видимости вычислений и позволяет использовать одинаковые имена переменных в различных областях видимости --- её наличие только на самом верхнем уровне означает, что все переменные принадлежат одной области видимости.
    \item Отсутствие унификаций термов-конструкторов позволяет не производить очевидной унификации в процессе выполнения алгоритма.
    \item Отсутствие вызовов на термах-конструкторах позволяет избежать неопределённости в процессе аннотирования.
\end{itemize}

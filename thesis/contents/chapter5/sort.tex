\subsection{Трансляция реляционной сортировки}
\label{lab:sort}

В этом разделе мы продемонстрируем, что описанный транслятор может транслировать программы на \miniKanren{} в более производительные функциональные программы. 
Для демонстрации было выбрано отношение $permSort$. 
Его код в конкретном синтаксисе \miniKanren{} можно приведен на рисунке~\ref{lst:sort}.
Это отношение связывает список чисел Пеано с его отсортированной версией. 
Данное отношение можно использовать как для сортировки списка (запрос $permSort \ l \ ?$), так и для порождения всех возможных перестановок элементов этого списка (запрос $permSort \ ? \ sorted$). 

\begin{figure}[h!]
  \begin{center}
  \begin{minipage}{0.8\textwidth}
  \begin{lstlisting}[language=Haskell, frame=single, numbers=left,numberstyle=\small, firstnumber=193, escapechar=|]
:: permSort l sorted = conde
  (l === [] /\ sorted === [])
  ([small s s1:
     sorted === small % s1 /\ {smallesto l small s} /\ {permSort s s1}])

:: smallesto l small restl = conde
  (l === small % [] /\ restl === [])
  ([h t rh rt small1:
     l === h % t /\ {smallesto t small1 rt} /\ {minmaxo h small1 small rh} /\ restl === rh % rt])

:: minmaxo a b min max = conde
  ({leo a b trueo} /\ min === a /\ max === b)
  ({leo a b falseo} /\ min === b /\ max === a)

:: leo x y b = conde
  (x === zero /\ b === <trueo>)
  ([zz:
    x === succ zz /\ y === zero /\ b === falseo])
  ([x1 y1:
    x === succ x1 /\ y === succ y1 /\ {leo x1 y1 b})
  \end{lstlisting}
  \end{minipage}
  \end{center}
  \caption{Отношения, необходимые для выполнения сортировки $permSort$}
  \label{lst:sort}
\end{figure}

Данная сортировка является поистинне реляционной, однако ни при каком порядке конъюнктов не удается обеспечить вычисление результатов в обоих направлениях за обозримое время.
Так, реализация сортировки на \ocanren{}\footnote{Реализация отображения $permSort$:~\url{https://github.com/JetBrains-Research/OCanren/blob/60efb4d9d3b673d7162ffc95f159c16c5b80f3f4/samples/sorting.ml}, дата последнего доступа: 20.05.2020} способна вычислить результат только на списках длины не больше $6$, при этом в зависимости от порядка конъюнктов либо работает одно направление, либо другое. 
Результат трансляции $permSort \ l \ ?$ успешно сортирует списки из сотен элементов, а результат трансляции $permSort \ ? \ sorted$ генерирует все перестановки для списков длины $9$.
Завершения генерации перестановок списка длины $10$ дождаться не удалось. 
Стоит отметить, что функция генерации перестановок $permutations$ из стандартной библиотеки языка \haskell{}\footnote{Функция $permutations$ из стандартной библиотеки языка \haskell{}:~\url{https://hackage.haskell.org/package/base-4.14.0.0/docs/Data-List.html\#v:permutations}, дата последнего доступа: 20.05.2020}, примененная к списку длины $10$, не завершается за 5 минут.

Время работы транслированных программ в обоих направлениях приведено в таблице~\ref{tab:sort}. 
Все замеры производились на компьютере с процессором с характеристиками: Intel(R) Core(TM) i7-8550U CPU @ 1.80GHz.
Для каждого направления и каждой длины списков вычисление выполнялось 10 раз и усреднялось.

\begin{figure}[h!]
\begin{center}
\begin{tabular}{|r|c|c|c|c||c|c|}
      \hline
      направление & \multicolumn{4}{|c||}{$permSort \ l \ ?$} & \multicolumn{2}{|c|}{$permSort \ ? \ sorted$}  \tabularnewline
      \hline 
длина списка & 6 & 10 & 100 & 1000 & 6 & 9 \\ \hline
время в секундах & 0.106 & 0.124 & 0.162 & 5.284 & 0.124 & 0.964 \\
      \hline
\end{tabular}
\caption{Время работы результата трансляции $permSort$}
\label{tab:sort}
\end{center}
\end{figure}

Данный пример демонстрирует, что транслятор способен транслировать программы на \miniKanren{} в более производительные функциональные программы. 
Однако при работе над этим примером была выявлена проблема, требующая дальнейшего исследования.
Описанию этой проблемы посвящен следующий раздел.

\subsubsection{Проблема порядка конъюнктов при трансляции}
\label{lab:sortProblem}

Для двух разных направлений исходные программы отличались в порядке двух вызовов во втором дизъюнкте. 
Для $permSort \ l \ ?$ был выбран порядок, при котором сначала вызывается отношение $smallesto$, затем $permSort$ ($smallesto-permSort$).
Для $permSort \ ? \ sorted$ обратный: $permSort-smallesto$.
Если же выбрать для $permSort \ ? \ sorted$ порядок вызовов отношений такой же, что и для $permSort \ l \ ?$, программа не сможет выдать ни одного результата: код сгенерированной программы представлен на рисунке~\ref{lst:part}.

\begin{figure}[h!]
  \begin{center}
  \begin{minipage}{0.55\textwidth}
  \begin{lstlisting}[language=Haskell, frame=single, numbers=left,numberstyle=\small, firstnumber=193, escapechar=|]
  permSortOI1 s1$@$(s2 : s4) = do
  (s0, s3) <- smallestoOIO s2
  (c0) <- permSortOI s4
  if (s3 == c0) then return $\$$ (s0) else []        |\label{line:part4}|
  \end{lstlisting}
  \end{minipage}
  \end{center}
  \caption{Часть $permSortOI$, полученная при трансляции $permSort$ с последовательностью вызовов $smallesto-permSort$}
  \label{lst:part}
\end{figure}

Видно, что направление $smallesto$ выбрано не совсем удачно: данное отношение будет вынуждено генерировать списки.
В следующей строке $permSort$ также будет вынуждено генерировать списки.
В строке~\ref{line:part4} сгенерированные в двух предыдущих строчках списки будут фильтроваться.
Такое сочетание направлений делает невозможным выполнение $permSort$ в направлении $permSort \ ? \ sorted$.

В дальнейшем необходимо доработать транслятор, чтобы он, анализируя код программы на \miniKanren{} и направление трансляции выяснял бы наиболее удачную последовательность конъюнктов. 
Текущая реализация перебирает все возможные перестановки вызовов в одном дизъюнкте.
Успешной считается первая последовательность, при которой успешно завершилось аннотирование.
Соответственно, мы не можем гарантировать, что первым будет наилучшая из программ с точки зрения производительности. 


\documentclass[conference,a4paper,american]{IEEEtran}
\usepackage{listings}
\usepackage{tikz}
\usepackage{amsmath,amssymb,amsfonts}
\usepackage{algorithmic}
\usepackage{graphicx}
\usepackage[T2A]{fontenc}
\usepackage[utf8]{inputenc}
\usepackage{babel}
\usepackage{hyperref}
\usepackage{balance}

\newcommand{\miniKanren}{\textsc{miniKanren}}

\begin{document}

\title{Binding-Time Analysis for Relational Programs}

\author{
\IEEEauthorblockN{Irina Artemeva}
\textit{ITMO University} \\
Saint Petersburg, Russia \\
irinapluralia@gmail.com
\and
\IEEEauthorblockN{Ekaterina Verbitskaia}
\textit{JetBrains Research}\\
Saint Petersburg, Russia \\
kajigor@gmail.com
}


\maketitle

\begin{abstract}
Programs in relational programming are mathematical relations.
Such relations can be run in different directions: by providing some arguments of a program, one can compute the values of the others.
The execution of a program in the given direction is not always efficient.
One way to improve the performance of a relational program is to convert it into a functional program.
To create a function by a relation, it is necessary to determine the order in which names within the input program are bound with respect to the given direction.
Binding-time analysis is used to solve this problem in the area of program specialization, but it has not been created for relational programming before.
In this paper we propose a binding-time analysis algorithm for the relational programming language \miniKanren{}.
\end{abstract}

\begin{IEEEkeywords}
Relational programming, binding-time analysis, static analysis
\end{IEEEkeywords}

\end{document}

